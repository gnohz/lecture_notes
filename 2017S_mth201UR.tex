\documentclass[]{article}
\usepackage{etex}
\usepackage[margin = 1.5in]{geometry}
\setlength{\parindent}{0in}
\usepackage{amsmath}
\usepackage{amsfonts}
\usepackage{amssymb}
\usepackage{amsthm}
\usepackage{mathtools}
\usepackage{pgfplots}
\usepackage[lined]{algorithm2e}
\usepackage{qtree}
\usepackage{xytree}
\usepackage{float}
\usepackage[T1]{fontenc}
\usepackage{ae,aecompl}
\usepackage[pdftex, pdfproducer={LaTeX}, pdfcreator={pdflatex}]{hyperref}

\usepackage{cleveref}

% \newtheorem*{theorem}{Theorem}
% \theoremstyle{definition}
% \newtheorem*{definition}{Definition}
% \newtheorem{problem}{Problem}[section]
% \newtheorem{example}{Example}[section]

% \newtheorem*{lemma}[theorem]{Lemma}
% \newtheorem*{assumption}[theorem]{Assumption}
% \newtheorem*{proposition}[theorem]{Proposition}
% \newtheorem*{corollary}[theorem]{Corollary}
% \newtheorem*{remark}[theorem]{Remark}
\newtheorem{theorem}{Theorem}
\newtheorem{corollary}{Corollary}
\newtheorem{proposition}{Proposition}
\newtheorem{remark}{Remark}

\theoremstyle{definition}
\newtheorem{problem}{Problem}
\newtheorem{definition}{Definition}
\newtheorem{example}{Example}

\numberwithin{equation}{section}
\numberwithin{theorem}{section}


\setlength{\marginparwidth}{1.5in}
\setlength{\algomargin}{0.75em}

\DeclarePairedDelimiter{\set}{\lbrace}{\rbrace}

\definecolor{darkish-blue}{RGB}{25,103,185}

\usepackage{hyperref}
\hypersetup{
    colorlinks,
    citecolor=darkish-blue,
    filecolor=darkish-blue,
    linkcolor=darkish-blue,
    urlcolor=darkish-blue
}
\newcommand{\lecture}[1]{\marginpar{{\footnotesize $\leftarrow$ \underline{#1}}}}


\newcommand{\cV}{\mathcal{V}}
\newcommand{\cU}{\mathcal{U}}
\newcommand{\cA}{\mathcal{A}}
\newcommand{\cM}{\mathcal{M}}
\newcommand{\cD}{\mathcal{D}}
\newcommand{\cE}{\mathcal{E}}
\newcommand{\cL}{\mathcal{L}}
\newcommand{\cJ}{\mathcal{J}}
\newcommand{\cP}{\mathcal{P}}
\newcommand{\cF}{\mathcal{F}}
\newcommand{\cH}{\mathcal{H}}
\newcommand{\cS}{\mathcal{S}}
\newcommand{\C}{\mathbb{C}}
\newcommand{\bH}{\mathbb{H}}
\newcommand{\bL}{\mathbb{L}}
\newcommand{\E}{\mathbb{E}}
\newcommand{\R}{\mathbb{R}}
\newcommand{\F}{\mathbb{F}}
\newcommand{\V}{\mathbb{V}}
\newcommand{\N}{\mathbb{N}}
\newcommand{\bP}{\mathbb{P}} % \P is used by the system

\newcommand{\wt}{\widetilde}
\newcommand{\ds}{\displaystyle}
\newcommand{\Ito}{It\^{o}}
\newcommand{\e}{\varepsilon}
\newcommand{\ol}{\overline}

\makeatletter
\def\blfootnote{\gdef\@thefnmark{}\@footnotetext}
\makeatother

\begin{document}
	% \let\ref\Cref
	\title{\bf{MTH 201: Introduction to Probability}}
	\date{Spring 2017, University of Rochester}
	\author{Jie Zhong}

	\blfootnote{I'd love to hear your feedback. Feel free to email me at \href{mailto:jiezhongmath@gmail.com}{jiezhongmath@gmail.com}.}

	\blfootnote{% See \href{http://cthomson.ca/notes}{cthomson.ca/notes} for updates.
		% \ifdefined\sha % Also, \commitDateTime should be defined.
		Last modified: \today 
    % \commitDateTime{} ({\href{https://github.com/christhomson/lecture-notes/commit/\sha}{\sha}}).
		% \fi
	}

	\maketitle
	\newpage
	\tableofcontents
	\newpage


%%%%%%%%%%%%%%%%%%%%%%%%%%%%%%%%%%%%%%%%%%%%%%%%%%%
\section{Introduction and sets} \lecture{January 18, 2017}
  \subsection{Course structure}
  \label{subsec:course}
  \begin{itemize}
  \item Instructor information.
  \item Course website: \href{https://web.math.rochester.edu/courses/current/201/}{https://web.math.rochester.edu/courses/current/201/}
  \item Textbook: on Blackboard.
  \item Grade: $10\%$ Webwork, $10\%$ Written Homework, $20\%$ First Midterm,
    $20\%$ Second Midterm, $40\%$ Final.
  \item Webwork: due every Friday at $11:59$ pm. Set $00$ due Friday, Jan 27;
    Set $1$ due Friday, Feb 3.
  \item Written Homework: $2$ or $3$ problems, due in class on Wednesdays. The
    first one is due Wednesdays, Feb 1. Write down your name and student ID, and
    staple it if it has more than one page.
  \item No late submissions! No makeup for exam!
  \item Old exams.
  \item Office hours: $4:30-5:30$ pm on MWF at Hylan $1008$.
  \end{itemize}
  \subsection{Objective and expectation}
  \label{subsec:objective}
  \textbf{Objective}

  Learn how to describe uncertainty in term of mathematical models and develop
  the skill of probabilistic reasoning.

  \textbf{What to expect}

  Math takes deliberate practice. You will need to do a lot of practice problems
  to do well in this course.
  
  When studying for exams, start by doing the practice exams and redoing the
  written homework and Webwork. But don't stop there. Redo the problems again
  until the process of solving them becomes automatic. This is how you'll know
  that you've mastered the material. To use your time efficiently, focus your
  practice on the things you haven't mastered.
  
  You are expected to read the textbook and review your class notes while doing homework and studying for exams.

  This is not a calculus class; we will not just be computing things according
  to simple algorithms. This class is on a higher level. Logical reasoning will
  be very important. I will show you the fundamental concepts and techniques.
  But you will often be asked to combine these in new ways in the homework and
  on exams.

  In class, I will describe the definitions and give simple instructive examples and proofs. We want to nail the basics, because that's the foundation on which you'll build.

  Probability theory is built on set theory, so that's where we'll begin....

  \subsection{Sets}
  \label{subsec:sets}
  \begin{definition}[Sets]
    A \textbf{set} is a collection of objects, usually numbers. The objects in
    the set are called \textbf{elements} of the set.
  \end{definition}

  \begin{example}
    $A = \{1, 2,3\}$.

    $1\in A$ ($1$ is an element of $A$).

    $4 \notin A$ ($4$ is not an element of $A$).
  \end{example}

  \begin{definition}
    We write

    $\N = \text{set of positive integers} = \{1, 2, 3,\cdots\}$;

    $\mathbb{Z} = \text{set of integers} = \{0, 1, -1, 2, -2,\cdots\}$;

    $\R = \text{set of real numbers}$;

    $\varnothing = \text{empty set = set of no elements} = \{\}$.
  \end{definition}

  There is another way to describe a set:
  \[
    \{x \mid x~ \text{satisfies}~ P\} = \text{set of all elements having property}~
    P.
  \]

  \begin{example}
    \[
      [0,1] = \{x \in\R\mid 0 \le x \le 1\};
    \]
    \[
      A = \{n\in \N\mid n~ \text{is a square of a positive integer}\} = \{n^2\mid n\in \N\} = \{1 ,4,
      9, 16, 25, \cdots\}.
    \]
  \end{example}

  \begin{definition}[Subsets]
    $A$ is subset of $B$ if every element of $A$ is an element of $B$, and we
    write $A \subseteq B$.
  \end{definition}

  {\color{red} insert a venn's graph here!} 

  \begin{remark}
    $A = B$ if and only if $A\subseteq B$ and $B\subseteq A$, if and only if $A$
    and $B$ have the same elements.

    The order in the set does not matter. $\{1,2,3\} = \{2, 3, 1\}$.
  \end{remark}
  \begin{example}
    $A = \{1, 2, 3\}$, $B= \{1, 2, 3, 4, 5, 6\}$, and $C = \{7\}$. Then we have
    \[
      A \subseteq B, A\nsubseteq C, B\nsubseteq A.
    \]
  \end{example}

  \begin{definition}
    [Union and Intersection]
    \[
      A \cup B = \text{union of } A~\text{and}~B = \text{set of elements that
        belong to } A~\text{or}~B = \{x\mid x\in A~\text{{\color{red} or}}~x\in B\};
    \]
    \[
      A \cap B = \text{intersection of } A~\text{and}~B = \text{set of elements that
        belong to both } A~\text{and}~B = \{x\mid x\in A~\text{{\color{red} and} }~x\in B\}.
    \]
  \end{definition}

%   \def \setA{ (0,0) circle (1cm) }
%   \def \setB{ (1.5,0) circle (1cm) }
%   \def \myrectangle{ (-2, -1.5) rectangle (3.5, 1.5) }
%   \begin{center}
%     \begin{tikzpicture}
%       \draw \myrectangle node[below left]{$\Omega$};
%       \begin{scope}
%         \clip \setA ;
%         \fill[gray] \setB ;
%       \end{scope}
%       % start of clip scope
%       % end of clip scope
%       \draw \setA node[left] {$A$};
%       \draw \setB node[right] {$B$};
%     \end{tikzpicture}

%     \begin{tikzpicture}
% \draw (-2,-1.5) rectangle (3,1.5) node[below left]{$U$}; \fill[gray] (0,0) circle (1cm);
% \fill[gray] (1,0) circle (1cm);
% \draw (0,0) circle (1cm);
% \draw (1,0) circle (1cm); \draw (-1,1) node {$A$}; \draw (3,1) node {$B$};
% \end{tikzpicture}
%   \end{center}

 {\color{red} insert venn's graphs here!} 

  \begin{example}
    $A = \{1, 2, 3\}$ and $B = \{3, 4\}$, then $A\cup B = \{1, 2, 3, 4\}$ and
    $A\cap B = \{3\}$.
  \end{example}
  
  \begin{definition}
    [Union and Intersection of Many Sets]
    Let $A_1, A_2, \cdots$ be sets.
    \begin{align*}
      \bigcup_{i=1}^n A_i & = A_1\cup A_2\cup\cdots\cup A_n\\
      & = \{x\mid x\in
        A_1~\text{or}~x\in A_2~\text{or}~\cdots~\text{or}~x\in A_n\}\\
                          & = \{x\mid x\in A_i~\text{for some}~i\in \{1, \cdots,n\}\},
    \end{align*}
    \begin{align*}
      \bigcup_{i=1}^\infty A_i & = \bigcup_{i\in \mathbb{N}} A_i = \{x\mid x\in A_i~\text{for some}~i \in \N\},
    \end{align*}
    where ``some'' means ``as least one''.
    \begin{align*}
          \bigcap_{i=1}^n A_i & =A_1\cap A_2\cap\cdots\cap A_n\\
      & = \{x\mid x\in
        A_1~\text{and}~x\in A_2~\text{and}~\cdots~\text{and}~x\in A_n\}\\
                          & = \{x\mid x\in A_i~\text{for all}~i\in \{1, \cdots,n\}\},
    \end{align*}
    \begin{align*}
      \bigcap_{i=1}^\infty A_i & = \bigcap_{i\in \mathbb{N}} A_i = \{x\mid x\in A_i~\text{for all}~i \in \N\}.
    \end{align*}
  \end{definition}

  \begin{example}
    Let $A_1 = \{1\}, A_2 = \{1,2\}, A_3 = \{1,2,3\},\cdots$. Then
    \[
      \bigcup_{i=1}^n A_i = \{1,\cdots, n\}, \quad \bigcup_{i=1}^\infty A_i =
      \{1, 2, 3,\cdots\} = \N,
    \]
    and
    \[
      \bigcap_{i=1}^\infty A_i = \{1\},\quad \bigcap_{i=5}^{10} = A_5\cap A_6
      \cap\cdots\cap A_{10} = \{1, 2, 3, 4,5\}.
    \]
  \end{example}

  \begin{definition}
    [Disjoint Sets]
    Sets $A$ and $B$ are called \textbf{disjoint} (mutually exclusive) if $A\cap
    B = \varnothing$.

    Sets $A_1, A_2,\cdots$ are called \textbf{disjoint} (mutually exclusive) if
    $A_i\cap A_j = \varnothing$ for each pair $i, j$ with $i\neq j$.
  \end{definition}

  \begin{example}
    Let $A=\{1,2,3\}, B=\{7, 22,45\}$. Then $A\cap B =\varnothing$, and thus $A$
    and $B$ are disjoint.

    $A_1 =\{1\}, A_2=\{2\}, A_3 = \{3\},\cdots$, then $A_1, A_2,\cdots$ are disjoint.
  \end{example}

  
  \begin{definition}
    [Universe, Complement and Difference]\lecture{January 23, 2017}
    Let $\Omega$ be a \textbf{universe} set, i.e., a set that contains all the
    objects of interest in a particular context. Let $A$ and $B$ be subsets of $\Omega$.
    \begin{itemize}
    \item \textbf{Complement}: $A^c = \{x\in \Omega\mid x\notin A\}$.
    
    {\color{red} insert a venns' graph here!}
    \item \textbf{Difference}: $A\setminus B =  A - B = \{x\in\Omega\mid x\in A
    ~\text{\color{red} and}~x\notin B \} = A\cap B^c$.
    
    {\color{red} venn's graph here!} 
    \end{itemize}
  \end{definition}

  \begin{example}
    Let $\Omega = \R$, $A = [0,1] = \{x\in \R: 0\le x\le 1\}$, and $B =
    \mathbb{Z}$. Then
    \[
      A^c = (-\infty, 0)\cup (1,\infty),\quad A\setminus B = (0,1).
    \]
  \end{example}

  \textbf{Pop Quiz}: 
  \begin{itemize}
  \item $\Omega^c = \varnothing$, $\varnothing^c = \Omega$.
  \item What is $\bigcup_{n\in \mathbb{Z}} (n,n+1)$?

  Answer: $\{x\in \R\mid x\notin \Z\} = \R \setminus \Z$.
  \end{itemize}

  \textbf{Algebra of Sets}:
  \begin{itemize}
  \item $(A^c)^c = A$;
  \item $A\cap A^c = \varnothing$;
  \item $A\cup \Omega = \Omega$;
  \item $A\cap \Omega = A$.
  \end{itemize}

  \textbf{De Morgan's Laws}:
  \begin{itemize}
  \item $(A\cup B)^c = A^c \cap B^c$;
  \item $\left( \bigcup_i A_i \right)^c = \bigcap_i A_i^c$;
  \item $(A\cap B)^c = A^c \cup B^c$;
  \item $\left( \bigcap_i A_i  \right)^c = \bigcup_i A_i^c$.
  \end{itemize}
  \begin{proof}
    \begin{align*}
      x\in (A\cup B)^c & \Leftrightarrow x\notin A\cup B \Leftrightarrow x\notin A~\text{and}~x\notin B\\
                       & \Leftrightarrow x\in A^c \cap B^c \quad (\text{``$\Leftrightarrow$'' means ``if and only if'' means ``is equivalent to''}).
    \end{align*}

    Rest of proof: Exercise.
  \end{proof}

  \textbf{Distributive Properties}:
  \begin{itemize}
  \item $A\cap (B\cup C) = (A\cap B) \cup (A\cap C)$;
  \item $A\cup (B\cap C) = (A\cup B)\cap (A\cup C)$.
  \end{itemize}
  
%%%%%%%%%%%%%%%%%%%%%%%%%%%%%%%%%%%%%%%%%%%%%%%%%%%

\section{Random outcomes and random variables }
\label{sec:rand-outc-rand}
  \subsection{Sample spaces and probabilities}
  \label{subsec:1.1}
  \begin{definition}
    A probability model is a mathematical description of an uncertain situation.
    It has two parts:
    \begin{enumerate}
    \item The \textbf{sample space} $\Omega$, i.e., the set of all possible
      outcomes of an experiment.
      \begin{itemize}
      \item Elements of $\Omega$ are \textbf{outcomes} (also called
        \textbf{sample points}), denoted by $\omega$.
      \item Subsets of $\Omega$ are called \textbf{events}.
      \item The set of all events is denoted by $\cF$.
      \end{itemize}
    \item The \textbf{probability measure} $\bP$.
      \begin{itemize}
      \item $\bP$ is a function from $\cF$ to $\R$;
      \item To each event $A$, it assigns a number $\bP(A)$, called
        the \textbf{probability} of $A$.
      \item The probability measure $\bP$ is sometimes called a probability law or
        probability distribution.
      \end{itemize}
    \end{enumerate}
    The triple $(\Omega, \cF, \bP)$ is called a \textbf{probability space}.
  \end{definition}
  \vspace{1em}
  \textbf{Axioms of Probability}:
  \begin{enumerate}
  \item $0\le \bP(A)\le 1$ for each event $A$;
  \item $\bP(\Omega) = 1$ and $\bP(\varnothing) =0$;
  \item If $A_1, A_2, \cdots$ is a sequence of (pairwise) disjoint events, then
    \begin{equation}
      \label{eq:additivity}
      \bP\left( \bigcup_i A_i \right) = \sum_i \bP(A_i).\quad(\text{Countable additivity})
    \end{equation}
  \end{enumerate}

  \begin{corollary}
    If $A_1, A_2, \cdots, A_n$ are disjoint events, then
    \begin{equation}
      \label{eq:add-finite}
      \bP(A_1\cup \cdots \cup A_n) = \bP(A_1) + \cdots + \bP(A_n).
    \end{equation}
  \end{corollary}

  \begin{proof}
    It is a direct consequence of~\eqref{eq:additivity} by setting $A_{n+1} =
    A_{n+2} =\cdots = \varnothing$.
  \end{proof}


  \subsection{Equally likely outcomes}
  \label{subsec:1.2}
  \textbf{Fact.} Suppose the sample space has finitely many equally likely
  outcomes, then for any event $A$,
  \[
    \bP(A) = \frac{\# A}{\# \Omega} = \frac{\text{number of elements in
        $A$}}{\text{number of elements in $\Omega$}}.
  \]
  \begin{proof}
    Say $\Omega$ has $n$ equally likely outcomes, then $\bP(\omega) = 1/n$ for
    every $\omega \in \Omega$.

    For any event $A = \{a_1, \cdots, a_k\}$,
    \[
      \bP(A) = \bP\left( \bigcup_{i=1}^k \{a_i\}\right) =
      \sum_{i=1}^k\bP(\{a_i\}) = \frac{k}{n},
    \]
    where the last equality is obtain by \eqref{eq:add-finite}.
  \end{proof}

  \begin{example}
    Roll two fair six sided dice, distinguish between them in some way: a first
    one and second one. One possible outcome is $(3,5)$.
    \begin{enumerate}
      [(1)]
    \item How many possible outcomes are there?
    \item How many ways to roll doubles?
    \item What is the probability of rolling doubles?
    \end{enumerate}
  \end{example}

  \textbf{Answer.} (1). There are $6$ possible outcomes for the first die. For
  each of these, there are $6$ possible outcomes for the second die. Then the
  total is $6\cdot 6 = 36$, i.e.,
  \[
    \Omega = \{(i,j)\mid i,j = 1,2,3,4,5,6\}.
  \]

  (2). For each outcome of the first die, there is exactly one possible outcome
  for the second die, because they have to match. So there are $6\cdot 1 = 6$
  ways to roll doubles.
  

  (3). The probability of rolling doubles is
  \[
    \bP = \frac{\# ~\text{outcomes with doubles}}{\#~\text{all possible
        outcomes}} = \frac{6}{36} = \frac{1}{6}.
  \]

  \begin{remark}
    When the sample space has finitely many equally likely outcomes, we can
    compute the probabilities by \textbf{counting}.
  \end{remark}

  \textbf{Counting Principle.} Consider a $r$-stage process. Suppose
  \begin{itemize}
  \item There are $n_1$ possible outcomes for the $1$st stage;
  \item For each possible outcome of the $1$st stage, there are $n_2$ possible
    outcomes for the $2$nd stage;
  \item For each possible outcome of the first two stages, there are $n_3$
    possible outcomes for the $3$rd stage, $\cdots$, and so on.
  \end{itemize}
  Then the total number of possible outcomes is
  \[
    n_1\cdot n_2\cdot n_3\cdots n_r.
  \]

  \begin{example}
    A telephone number is a $7$-digit sequence of numbers in $\{0,1,2,\cdots,
    9\}$, but the first digit cannot be zero or one. How many distinct telephone
    numbers are there?
  \end{example}

  \textbf{Answer.} This is a $7$ stage process. There are $8$ choices for the
  first stage (or first digit), and $10$ choices for each stage after. So the
  total number is
  \[
    8\cdot 10\cdot 10\cdot 10\cdot 10\cdot 10\cdot 10 = 8\cdot 10^6.
  \]

  \begin{example}
    A screen has $N$ pixels. Each pixel can be off (black) or on (white). How
    many possible images can the screen display?
  \end{example}

  \textbf{Answer.} This is an $N$ stage process. There are $2$ choices for each
  stages: pixel on or off. The total number is $2^N$.

  \textbf{Permutations.} How many ways are there to permute (means ``arrange'',
  or ``order'') $n$ distinct items?

  \textbf{Answer.} $n$ stage process. 
  \begin{itemize}
  \item $1$st stage: choose which item goes in position $1$. There are $n$ items
    to choose from.
  \item $2$nd stage: choose which item goes in position $2$. There are $n-1$
    items to choose from.
  \item $\cdots$
  \item $n$th stage: choose which item goes in position $n$. There
    is only one item to choose.
  \end{itemize}
  Thus, the total number of ways to permute $n$ distinct items is
  \[
    n(n-1)\cdots 1 = {\color{red} n!}.
  \]

  \begin{example}
    How many ways can we line up all $60$ students in this class?
  \end{example}
  \textbf{Answer.} $60\cdot 59\cdots 1 = 60! \approx 8\cdot 10^{81}$ (greater
  than the number of atoms in the observable universe).

  \textbf{Ordered Selection.} ($k$-permutations) It is a selection of items from
  a set such that the order we select them matters.

  How many ordered selection of $k$ items from a set of $n$ distinct items are
  possible?

  $k$ stage process:
  \begin{itemize}
  \item $1$st stage: choose which item goes in position one: $n$ choices.
  \item $2$nd stage: choose which item goes in position two: $n-1$ choices.
  \item $\cdots$
  \item $k$th stage: choose which item goes in position $k$: ${\color{red}
      n-k+1} $ choices.
  \end{itemize}

  So the total number is
  \[
    n\cdot n-1\cdots (n-k+1).
  \]

  \begin{remark}
   Permutation is a special ordered selection. 
  \end{remark}

  \begin{example}
    How many ways we can line up $5$ of the $60$ students in the class?
  \end{example}

  \textbf{Answer.} $60\cdot 59\cdot 58\cdot 57\cdot 56 = 655,381,440$.

  \textbf{Combinations.} (unordered selection) It is a selection of items from a
  set such that the order of the selection does not matter.

  How many combinations of $k$ items from a set of $n$ distinct items are
  possible?

  We call this number ``$n$~choose~$k$'', denoted by
  \[
    \binom{n}{k}.
  \]

  How to find this number? We will start from the ordered selection.
  
  Given a set of $n$ distinct items. Making an ordered selection of $k$ items is
  the same as choosing a combination of $k$ items and then ordering them.

  This is a $2$ stage process.

  The number of ordered selection of $k$ items is equal to the number of
  combinations of $k$ items from $n$ distinct items times the number of ways to
  order the $k$ items:
  \[
    n(n-1)\cdots(n-k+1) = \binom{n}{k}\cdot k!.
  \]
  Therefore, the number of combinations of $k$ items from a set of $n$ distinct
  items is
  \[
    \binom{n}{k} = \frac{n(n-1)\cdots (n-k+1)}{k!} = \frac{n!}{k!(n-k)!}.
  \]
  
  \begin{example}
    Select $5$ of $60$ students in class without regard to order.
  \end{example}
  \textbf{Answer.} $\ds\binom{60}{5}$, or
  \[
    \frac{60\cdot 59\cdot 58 \cdot 57 \cdot 56}{5\cdot 4 \cdot 3\cdot 2\cdot 1}
    = 5,461,512 = \frac{60!}{5!55!}.
  \]
  \textbf{Facts.}
  \begin{itemize}
  \item $\ds\binom{n}{k} = \binom{n}{n-k}$.
  \item $\ds\binom{n}{k}$ is also called a binomial coefficient because it is equal
    to the coefficient of $x^ky^{n-k}$ in the expansion
    \[
      (x+y)^n = \sum_{k=0}^n \binom{n}{k}x^k y^{n-k}.
    \]
  \end{itemize}

  \textbf{Partitions.} A combination is a choice of a $k$-element subset of an
  $n$-element set, and the order does not matter.

  This is the same as partitioning the set into two parts. One part contains the
  $k$ elements, and the other part contains the remaining $n-k$ elements.

  Now consider partitions into more than two parts.

  Given a set of $n$ distinct items and non-negative integers $n_1, n_2,\cdots,
  n_r$ with $n=n_1+n_2+\cdots n_r$. How many ways can the set be partitioned
  into $r$ disjoint subsets with $n_i$ items in the $i$th subset? We call this
  number $\ds \binom{n}{n_1,n_2,\cdots,n_r}$.

  This is a $r$ stage process:
  \begin{itemize}
  \item Stage $1$: to form the first subset, choose a combination of $n_1$ items
    out of the $n$-item set.
  \item Stage $2$: to form the second subset, choose a combination of $n_2$
    items out the remaining $n-n_1$ items.
  \item and so on...
  \end{itemize}

  The total number of partition is
  \begin{align*}
    &\binom{n}{n_1}\binom{n-n_1}{n_2}\binom{n-n_1-n_2}{n_3}\cdots\binom{n-n_1-\cdots
    - n_{r-1}}{n_r} \\
    =&\ \frac{n!}{n_1!(n-n_1)!}\cdot \frac{(n-n_1)!}{n_2!(n-n_1-n_2)!}\cdot \frac{(n-n_1-n_2)!}{n_3!(n-n_1-n_2-n_3)!}\cdots \frac{(n-n_1-\cdots -n_{r-1})!}{n_r!(n-n_1-\cdots - n_{r-1}-n_r)!}\\
    =&\ \frac{n!}{n_1!\cdot n_2!\cdots n_r!}.
  \end{align*}

  Therefore,
  \[
    \binom{n}{n_1,n_2,\cdots, n_r} = \frac{n!}{n_1!\cdot n_2!\cdots n_r!}.
  \]

  \begin{example}
    How many arrangements are there of the letters ``BANANA''? 
  \end{example}

  This is an example of arrangement with identical objects.

  \textbf{Solution 1.} There are $6$ positions for the letters. Each arrangement
  is a partition of the set of $6$ positions into a subset of size $3$ (the
  positions that get the letter ``A''), a subset of size $2$ (the positions that
  get the letter ``N'') and a subset of size $1$ (the position that gets letter
  ``B''). For example,
  \[
    \underset{1}{\text{A}}\underset{2}{\text{A}}\underset{3}{\text{A}}\underset{4}{\text{B}}
    \underset{5}{\text{N}} \underset{6}{\text{N}}.
  \]
  
  So the total number of arrangements is
  \[
    \binom{6}{3,2,1} = \frac{6!}{3!2!1!} = 60.
  \]

  
  \subsection{Infinitely many outcomes}
  \label{subsec:1.3}

  \subsection{Consequences of the rules of probability}
  \label{subsec:1.4}

  \subsection{Random variables: a first look}
  \label{subsec:1.5}


%%%%%%%%%%%%%%%%%%%%%%%%%%%%%%%%%%%%%%%%%%%%%%%%%%%

\section{Conditional probability and independence}
\label{sec:cond-prob-indep}

%%%%%%%%%%%%%%%%%%%%%%%%%%%%%%%%%%%%%%%%%%%%%%%%%%% 

\section{Random variables}
\label{sec:random-variables}

%%%%%%%%%%%%%%%%%%%%%%%%%%%%%%%%%%%%%%%%%%%%%%%%%%% 

\section{Approximation of the Binomial distribution}
\label{sec:appr-binom-distr}

%%%%%%%%%%%%%%%%%%%%%%%%%%%%%%%%%%%%%%%%%%%%%%%%%%%

\section{Transformation of random variables}
\label{sec:transf-rand-vari}

%%%%%%%%%%%%%%%%%%%%%%%%%%%%%%%%%%%%%%%%%%%%%%%%%%%

\section{Joint distribution of random variables}
\label{sec:joint-distr-rand}

%%%%%%%%%%%%%%%%%%%%%%%%%%%%%%%%%%%%%%%%%%%%%%%%%%%

\section{Sums and symmetry}
\label{sec:sums-symmetry}

%%%%%%%%%%%%%%%%%%%%%%%%%%%%%%%%%%%%%%%%%%%%%%%%%%%

\section{Expectation and variance in the multivariate setting}
\label{sec:expect-vari-mult}

%%%%%%%%%%%%%%%%%%%%%%%%%%%%%%%%%%%%%%%%%%%%%%%%%%%

\section{Tail bounds and limit theorems}
\label{sec:tail-bounds-limit}

%%%%%%%%%%%%%%%%%%%%%%%%%%%%%%%%%%%%%%%%%%%%%%%%%%%

\section{Conditional distribution}
\label{sec:cond-distr}





\end{document}